\documentclass{article}
\usepackage[margin=1.25in]{geometry}
\usepackage{amsmath}
\usepackage{amssymb}
\usepackage{enumitem}
\setlist{nosep}
\usepackage{tcolorbox}
\usepackage[per-mode=fraction]{siunitx}
\usepackage{tikz}

\usetikzlibrary { decorations.pathmorphing, decorations.pathreplacing, decorations.shapes,
  angles, quotes}




\title{Homogeneous Heat Equation -- The 1D Rod with Dirichelet BCs and a uniform initial temperature} 
\author{SF Wolf}
\date{\today}

\begin{document}
\maketitle

\section{Problem setup}

Model a 1D object that has a length of 1, with a uniform initial temperature $T_0$. Each end of
the rod will be held at a temperature of 0. Assume no heat escapes along the length of the
rod. There are no external heat sources or sinks.

We can describe this system by the following set of equations:
\begin{align}
  \frac{\partial u(x,t)}{\partial t} - \frac{\partial^2 u(x,t)}{\partial x^2} = 0
  & \qquad x\in[0,1], t>0 \label{eq:pde}\\
  u(0,t) = u(1,t) = 0   & \qquad t>0 \label{eq:bc}\\
  u(x,0) = T_0   & \qquad x\in[0,1] \label{eq:ic}
\end{align}

\subsection{Step 1: Convert PDE to ODEs}
Our first step is to convert our partial differential equation (\ref{eq:pde}) into multiple
ordinary differential equations. We will apply the \textit{separation of variables} technique:
\begin{equation}
  \label{eq:sepVar}
  u(x,t) = X(x) T(t)
\end{equation}
Inserting this definition into the above, we get:
\[
  X(x)\frac{dT(t)}{dt} = \frac{d^2X(x)}{dx^2} T(t)
\]
Divide by $u(x,t) = X(x) T(t)$ and we get:
\[
  \frac{1}{T(t)}\frac{dT(t)}{dt} = \frac{1}{X(x)} \frac{d^2X(x)}{dx^2}
\]
This expression has the form
\[
  \text{Stuff that only depends on } t = \text{Stuff that only depends on } x
\]
So both sides must be constant. This allows us to write:
\begin{align}
  \frac{dT(t)}{dt} &= -\lambda T(t) \label{eq:ode1} \\
  \frac{d^2X(x)}{dx^2} &= -\lambda X(x) \label{eq:ode2}
\end{align}

\paragraph{Revised Boundary condition}
If we apply separation of variables to the boundary condition in equation (\ref{eq:bc}), we
get:
\[
  0 = u(0,t) = X(0) T(t)
\]
and
\[
  0 = u(1,t) = X(1) T(t)
\]
Assuming that the function $T(t)\neq 0$, then we can write
\begin{equation}
  \label{eq:newbc}
  X(0) = X(1) = 0
\end{equation}

\subsection{Step 2: Get a general solution for $u(x,t)$}
Next, we will get a general solution for $u(x,t)$. We will do this by considering 3 cases:
\begin{enumerate}
  \item $\lambda = 0$
  \item $\lambda > 0$
  \item $\lambda < 0$
\end{enumerate}
and apply the boundary conditions in equation (\ref{eq:newbc}) to each case

\subsubsection{Case 1: $\lambda = 0$}
If $\lambda = 0$, then we can write equation (\ref{eq:ode2}) as:
\[
  \frac{d^2X(x)}{dx^2} = 0 \implies X(x) = ax+b
\]
If we use the boundary condition in equation (\ref{eq:newbc}) to find $a$ and $b$, we get
\[
  a = b = 0 \implies X(x) = 0
\]
This is what math texts would call the trivial solution, and I call BORING! So we throw that
solution out.

\subsubsection{Case 2: $\lambda > 0$}
If $\lambda > 0$, then we can define a $k>0$ such that $\lambda = k^2$ and then we
can write equation (\ref{eq:ode2}) as:
\[
  \frac{d^2X(x)}{dx^2} = -k^2 X(x) \implies X(x) = a\sin(kx) + b\cos(kx)
\]
If we use the boundary condition at $x=0$ in equation (\ref{eq:newbc}) to find $a$, $b$, and
$k$ we get
\[
  X(0) = 0 = a\sin 0 + b \cos 0 \implies b=0 \implies X(x) = a \sin(kx)
\]
Now apply the boundary condition at $x=1$:
\[
  X(1) = 0 = a \sin k \implies a = 0 \text{ or } \sin k=0
\]
Allowing $a=0$ is boring--as defined above, so let's find possible values for $k$:
\[
  \sin k = 0 \implies k = n\pi, n\in\mathbb{N} \qquad (n=1, 2, 3, \ldots)
\]
So we know there are infinitely many solutions of the form
\[
  X(x) = a \sin(n\pi x) \quad \lambda = n^2\pi^2 \quad n\in\mathbb{N}
\]

\subsubsection{Case 3: $\lambda < 0$}
If $\lambda < 0$, then we can define a $\kappa>0$ such that $\lambda = -\kappa^2$ and then we
can write equation (\ref{eq:ode2}) as:
\[
  \frac{d^2X(x)}{dx^2} = \kappa^2 X(x) \implies X(x) = a\sinh(\kappa x) + b\cosh(\kappa x)
\]
If we use the boundary condition at $x=0$ in equation (\ref{eq:newbc}) to find $a$, $b$, and
$k$ we get
\[
  X(0) = 0 = a\sinh 0 + b \cosh 0 \implies b=0 \implies X(x) = a \sinh(\kappa x)
\]
Now apply the boundary condition at $x=1$:
\[
  X(1) = 0 = a \sinh \kappa \implies a = 0 \text{ or } \sinh \kappa=0
\]
Allowing $a=0$ is boring--as defined above, however, $\sinh \kappa = 0$ only when $\kappa=0$
which is also not allowed. So this case has no meaningful solutions as well.

Since we need to include all possible solutions in our general solution, we can write:

\begin{align}
  \label{eq:genSol}
  u(x,t) &= \sum_{n=1}^{\infty} X_n(x) T_n(t) \nonumber \\
   &= \sum_{n=1}^{\infty} a_n \sin(n\pi x) e^{-n^2\pi^2t}
\end{align}

\subsection{Step 3: Find our particular solution for $u(x,t)$}
Use the initial condition in equation (\ref{eq:ic}) to find our particular solution for
$u(x,t)$, specifically, we can say:
\[
  T_0 = \sum_{n=1}^{\infty} a_n \sin(n\pi x)
\]
We can get the constants $a_n$ by integrating the above equation as follows
\[
  \int_0^1 dx \sin(m\pi x) T_0 = 
  \int_0^1 dx \sin(m\pi x) \sum_{n=1}^{\infty} a_n \sin(n\pi x)
\]
Solving the integrals, we find:
\[
  a_n = 2T_0 \int_0^1 \sin(n\pi x) =
  \begin{cases}
    0 & n=2,4,6,\ldots \\
    \frac{4T_0}{n\pi} & n=1,3,5,\ldots
  \end{cases}
\]
Therefore:
\begin{equation}
  \label{eq:finalSoln}
  u(x,t) = \sum_{n=1}^{\infty} \frac{4T_0}{(2n-1)\pi}\sin\left((2n-1)\pi x\right) e^{-(2n-1)^2\pi^2t}  
\end{equation}





\end{document}
